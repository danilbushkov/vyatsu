\subsection{Разработка структуры клиентской части конструктора}


В данном разделе описываются разработанные структуры клиентской части
конструктора и его диаграммы состояний.

\subsubsection{Структура интерфейса конструктора}

Пользовательский интерфейс конструктора представляет собой
административную панель с набором следующих страниц:

\begin{itemize}
	\item страница аутентификации;
	\item страница регистрации;
	\item страница ботов пользователя конструктора;
	\item страница создания нового бота;
	\item страница запуска бота;
	\item страница редактирования бота.
\end{itemize}


Каждая страница состоит из верхней панели и содержимого страницы.
Верхняя панель содержит ссылки на страницы аутентификации и регистрации,
если пользователь не вошёл в систему, иначе - кнопку “выйти из системы”.
Шаблон представлен на рисунке~\ref{f:general-template}.

\begin{figure}[ht]
	\centering
	\vspace{0.5cm}
	\includegraphics[width=0.6\textwidth]{page-templates/general}
	\caption{Общий шаблон страниц}
	\label{f:general-template}
\end{figure}


Страница аутентификации и регистрации содержат поля ввода логина и
пароля пользователя, под которым располагается кнопка входа или
регистрации. Шаблон содержимого страницы аутентификации представлен на
рисунке~\ref{f:auth-template}.

\begin{figure}[ht]
	\centering
	\vspace{0.5cm}
	\includegraphics[width=0.4\textwidth]{page-templates/auth}
	\caption{Шаблон содержимого страницы аутентификации}
	\label{f:auth-template}
\end{figure}


Страница ботов содержит список блочных элементов, которые
включают в себя:
\begin{itemize}
	\item название бота;
	\item статус бота;
	\item кнопка для перехода к редактированию бота;
	\item кнопка запуска или остановки бота.
\end{itemize}

Шаблон содержимого страницы списка ботов представлен на
рисунке~\ref{f:list-template}.

\begin{figure}[ht]
	\centering
	\vspace{0.5cm}
	\includegraphics[width=0.7\textwidth]{page-templates/list}
	\caption{Шаблон содержимого страницы списка ботов}
	\label{f:list-template}
\end{figure}

Страница создания бота содержит одно поле ввода, под которым
располагается кнопка создания. Шаблон содержимого представлен на
рисунке~\ref{f:form-template}.


\begin{figure}[ht]
	\centering
	\vspace{0.5cm}
	\includegraphics[width=0.4\textwidth]{page-templates/form}
	\caption{Шаблон содержимого страницы создания бота}
	\label{f:form-template}
	\vspace{0.5cm}
\end{figure}

Страница запуска бота включает в себя поле ввода токена и кнопку
запуска. Шаблон имеет такую же структуру, как и у содержимого страницы создания бота,
только с другим именованием кнопки и заголовка поля ввода.

Страница редактирования бота содержит визуальный редактор.

\subsubsection{Разработка структуры визуального редактора}

Визуальный редактор ботов представляет собой область, на которой
пользователь может добавлять, редактировать и удалять компоненты, а также
связывать их между собой.

\paragraph{Модульная структура редактора}


Редактор состоит из следующих модулей (Рисунок~\ref{f:mod-client-editor-struct}):

\begin{itemize}
	\item модуль Api-клиента;
	\item модуль контроллера;
	\item модуль представления;
	\item модуль хранилища.
\end{itemize}

\begin{figure}[ht]
	\centering
	\vspace{0.5cm}
	\includegraphics[width=0.7\textwidth]{structures/client-editor/mod}
	\caption{Модульная структура редактора}
	\label{f:mod-client-editor-struct}
\end{figure}


API клиент содержит в себе функции обращения к серверу конструктора
для получения и обновления данных бота.

Хранилище содержит методы для изменения данных редактора. При
изменении данных хранилища происходит обновление их и на сервере через

API клиент.
Контроллер служит посредником между представлением и хранилищем:
он содержит обработчики, которые меняют состояние редактора. Использует
функции API клиента.

Вид редактора (или представление) содержит в себе компоненты
редактора, от которых идут запросы от пользователя. Запросы передаются
контроллеру, который их обрабатывает.


\paragraph{Компонентная структура редактора}

Редактор можно разбить на иерархический набор компонентов: где
вышестоящий компонент является родителем, а компонент, который в нем
содержится, - дочерним.

Компоненты общаются друг с другом посредством передачи параметров
и вызова событий. Родительский компонент вызывает дочерний с помощью
передачи параметров, а также отлавливает события дочернего элемента при
изменении его состояния.

Компонентная структура визуального редактора представлена на
рисунке~\ref{f:comp-client-editor-struct}.

\begin{figure}[ht]
	\centering
	\vspace{0.5cm}
	\includegraphics[width=0.9\textwidth]{structures/client-editor/comp}
	\caption{Компонентная структура редактора}
	\label{f:comp-client-editor-struct}
\end{figure}

На самой высокой ступени стоит компонент редактор. Он хранит все
состояние приложения, а также изменяет его с помощью контроллера. Он
состоит из области редактора и панели добавления компонентов.

Панель компонентов содержит разные виды компонентов, которые
можно добавить на область редактора путем перетаскивания.

В области редактора содержится набор компонентов бота и их
соединений.

Компонент бота содержит в себе следующие компоненты:
\begin{itemize}
	\item содержимое компонента;
	\item входные точки компонента;
	\item выходные точки компонента;
	\item область соединения.
\end{itemize}

Компоненты включают в себя разные виды содержимого. Содержимое
зависит от типа компонента.

Чтобы контролировать переход по компонентам присутствуют
элементы соединений – точки соединения. Благодаря им можно располагать
линии между компонентами и тем самым связывать их.

Существует три вида точек соединений:
\begin{itemize}
	\item входная точка;
	\item выходная точка;
	\item временная точка.
\end{itemize}

Входная точка. Служит для обозначения места соединения у
следующего компонента. Может быть несколько – зависит от количества
предыдущих компонентов. При нажатии на данный элемент будет
происходить событие отвязки.

Выходная точка. Служит для указания следующего компонента. Может
быть несколько - зависит от типа компонента. При нажатии на точку будет
происходить событие начала соединения.

Временная точка соединения – элемент, который помогает
пользователю обозначить место соединения у следующего компонента.
Данная точка располагается на области соединения компонента. Вызывает
событие конца соединения при отжатии левой кнопки мыши на этом элементе.
При этом событии происходит скрытие временной и вставка входной точки.

Область соединения компонента представляет собой место, где
возможно расположение входных точек. Область охватывает края
компонента.

Также у каждого компонента присутствует кнопка удаления.

Расположение компонентов на шаблоне визуального редактора
представлено на рисунке~\ref{f:editor-template}.

\begin{figure}[ht]
	\centering
	\vspace{0.5cm}
	\includegraphics[width=0.9\textwidth]{page-templates/editor}
	\caption{Шаблон визуального редактора}
	\label{f:editor-template}
	\vspace{0.5cm}
\end{figure}


\subsubsection{Разработка диаграмм состояний клиентской части конструктора}

Диаграмма состояний клиентской части конструктора представлена на
рисунке~\ref{}.


\input{pages/content/struct/crd-calc}
