\subsection{Разработка структуры клиентской части конструктора}


В данном разделе описываются разработанные структуры клиентской части
конструктора и его диаграммы состояний.

\subsubsection{Структура интерфейса конструктора}

Пользовательский интерфейс конструктора представляет собой
административную панель с набором следующих страниц:

\begin{itemize}
	\item страница аутентификации;
	\item страница регистрации;
	\item страница ботов пользователя конструктора;
	\item страница создания нового бота;
	\item страница запуска бота;
	\item страница редактирования бота.
\end{itemize}


Каждая страница состоит из верхней панели и содержимого страницы.
Верхняя панель содержит ссылки на страницы аутентификации и регистрации,
если пользователь не вошёл в систему, иначе - кнопку “выйти из системы”.

Страница аутентификации и регистрации содержат поля ввода логина и
пароля пользователя, под которым располагается кнопка входа или
регистрации. Шаблон содержимого страницы аутентификации представлен на
рисунке

\paragraph{Длинный длинный длиннный длинный длинный длиннный длинный длинный длинный текст}
