\csection{Введение}

В современном мире стали популярными такие приложения для
быстрого общения как мессенджеры. Таких приложений достаточно много, но
большинство пользователей сети интернет все чаще отдают предпочтение
мессенджеру Telegram как наиболее удобному и надежному.

У Telegram имеется удобное API для создания ботов. Бот способен
выполнять определенные команды, заданные пользователем через интерфейс
Telegram. Данный функционал вполне может удовлетворять потребности
компании в предоставлении некоторых услуг в разных сферах. Например,
спортивные залы являются одной из таких сфер.

Создание ботов — это трудоемкий процесс, требующий
квалифицированных программистов, что довольно затратно для бизнеса.

Для решения данной проблемы существуют конструкторы Telegram-ботов, которые
предоставляют функции создания, редактирования и управления ботами.
К сожалению, большинство таких конструкторов предоставляют ограниченный функционал
при бесплатном использовании, а также имеют закрытые
способы хранения данных клиентов. Поэтому было принято решение выполнить анализ и
разработать конструктор Telegram-ботов без данных недостатков.



