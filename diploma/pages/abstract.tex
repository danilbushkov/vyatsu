
{

\topskip = 0.8cm
\begin{center}
	Реферат
\end{center}

\vspace{1em}

\authori\
\topic
:\
\mbox{\tpga}\
ВКР / ВятГУ, каф. ЭВМ; рук.
Долженкова М.Л. – Киров, \the\year. –
Гр.ч. 8 л. ф.А1;
ПЗ
\total{page} с.,
1 рис.,
1 табл.,
\total{equation} форм.,
1 источников,
1 прил.

\vspace{1.5em}

КОНСТРУКТОР,
TELEGRAM-БОТ,
КЛИЕНТСКАЯ ЧАСТЬ,
ВИЗУАЛЬНЫЙ РЕДАКТОР,
СЕРВЕРНАЯ ЧАСТЬ,
HTTP,
GOLAND,
POSTGRESQL,
TYPESCRIPT,
SOLID.JS,
JSON,
HTML,
CSS.

\vspace{1.5em}

Объект выпускной квалификационной работы - программное средство для упрощения
создания и управления ботами в мессенджере Telegram.

Целью данной выпускной квалификационной работы является повышение
скорости разработки, настройки и управления
ботами, что позволит пользователям
без специальных навыков
программирования создавать
эффективных ботов для различных целей.

Результат работы - конструктор Telegram-ботов,
который будет предоставлять
набор инструментов и
функций для создания и настройки
ботов, а также предоставлять
возможности для их управления.

}
